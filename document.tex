\RequirePackage[l2tabu,orthodox]{nag}
\documentclass{article}
% Uncomment to use the KOMA-Script scrartcl class. This allows things such as a subtitle.
% \documentclass{scrartcl}
% If you use the above, uncomment this to use the same font as the `article' class.
% \setkomafont{disposition}{\normalfont\bfseries}
\usepackage{amsmath}
\usepackage{microtype}
\usepackage{hyperref}
\hypersetup{hidelinks}

\newcommand{\mailtohref}[1]{\href{mailto:#1}{$<$\nolinkurl{#1}$>$}}
\newcommand{\floor}[1]{\left\lfloor#1\right\rfloor}
\newcommand{\ceil}[1]{\left\lceil#1\right\rceil}
\newcommand{\paren}[1]{\left(#1\right)}


\title{CIS 676 Database Performance Formula Sheet}
\author{Sean Fisk \mailtohref{sean@seanfisk.com} \\
  Cindy Vannoy \\
  Emily Johnson}

\begin{document}
\maketitle

\section*{Disclaimer}

These formulas may or may not be correct. They are personal
interpretations of formulas in the slides and book we are using.

\section{External Sorting}

\begin{align*}
  N &: \text{File size in pages} \\
  B_i &: \text{Number of Input Buffers} \\
  B_o &: \text{Number of Output Buffers} \\
  B_T &: \text{Total Buffer Pages Available} \\
  b &: \text{Number of blocks used for blocked I/O}
\end{align*}
\begin{align}
  \text{Number of Sorted Runs} &= \ceil{\frac{N}{B_T}}\\
  \text{Cost of Partial Sort (I/Os)} &= 2N
\end{align}

\subsection{Unblocked I/O}
\begin{align}
  \text{How many way merge} &= B_i \\
  \text{Number of Merge Passes} &= \ceil{\log_{B_i}\ceil{\frac{N}{B_T}}} \\
  \text{Cost of Merge Phase (I/Os)} &= 2N\ceil{\log_{B_i}\ceil{\frac{N}{B_T}}} \\
  \text{Total Cost of External Sort} &= 2N\paren{\ceil{\log_{B_i}\ceil{\frac{N}{B_T}}} + 1}
\end{align}

\subsection{Blocked I/O}
\begin{align}
  \text{How many way merge} &= \ceil{\frac{B_i}{b}} \\
  \text{Number of Merge Passes} &= \ceil{\log_{\ceil{\frac{B_i}{b}}}\ceil{\frac{N}{B_T}}} \\
  \text{Cost of Merge Phase (I/Os)} &= 2N\ceil{\log_{\ceil{\frac{B_i}{b}}}\ceil{\frac{N}{B_T}}} \\
  \text{Total Cost of External Sort} &= 2N\paren{\ceil{\log_{\ceil{\frac{B_i}{b}}}\ceil{\frac{N}{B_T}}} + 1}
\end{align}

\section{Relational Operations}

\begin{align*}
  B(R) &: \text{Size of Relation \(R\) in blocks} \\
  C(R) &: \text{Number of Tuples in Relation \(R\) (cardinality)}
\end{align*}

\subsection{Selection}

\subsubsection{Heap}

\textbf{Clarification:} \textit{Heap} just means a bunch of memory. It
does not refer to the \textit{heap data structure}. See Heap on Wikipedia for
more details.

\begin{align}
  \text{Equality on Candidate Key} &= \frac{B(R)}{2} \text{(scan until record is found)} \\
  \text{Equality on Non-Candidate Key} &= B(R) \text{(must scan until end of file)} \\
  \text{Range on Candidate Key} &= B(R) \\
  \text{Range on Non-Candidate Key} &= B(R)
\end{align}

\subsubsection{Sorted Relation (Binary Search)}

\begin{align}
  x &: \text{Number of Additional Pages Containing Qualifying Tuples} \notag \\
  \intertext{\(x\) is 0 if selecting on candidate key.}
  \text{Cost} &= \ceil{\log_2B(R)} + x
\end{align}

If given a percentage of qualifying tuples, multiply by the total pages in the relation (\(B(R)\)) to get \(x\).

\subsubsection{Clustered B+ Tree Index (alternative 1)}

\begin{align}
  x &: \text{Number of Additional Pages Containing Qualifying Tuples} \notag \\
  \intertext{\(x\) is 0 if selecting on candidate key.}
  \mathit{levels} &: Number of Levels in the B+ Tree \\
  \mathit{levels} &= \mathit{height} + 1 \\
  \text{Cost} &= \mathit{levels} + x
\end{align}

\end{document}
